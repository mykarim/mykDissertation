\chapter{Introduction}

Leaving aside the impetus of human curiosity for a moment, the ubiquity and commercial relevance of granular materials dictate that we understand the properties of these fascinating systems. From the sands on a beach \cite{schiffer_granular_2005} to the surface of Mars \cite{mars_and_moon}, from food grains to the rocky rings of Saturn \cite{brilliantov_size_2015, saturn_2017}, nature is full of these materials we describe as granular \cite{duran_sands_2000, jaeger_granular_1996}. 

The term granular materials describes a class of matter that has a characteristic energy scale far exceeding $k_{B}T$. Consequently, changes in ambient temperature does little to alter the properties of a granular system. So, for instance, changes in temperature along the Oregon coast does not result in a rearrangement of its sandy beaches. But temperature fluctuations do account for the complex and often unpredictable weather patterns experienced there. In other words, granular materials stand in stark contrast to gases and liquids where thermal effects lead to changes in the state of the system. Instead of $k_{B}T$, the characteristic energy scale in a granular system is set by $mgd$ where m is the mass of a constituent particle, g the acceleration due to gravity and d the characteristic size of the particle. Here, $mgd$ can be thought of as the energy required for one grain to move past another. 

Unfortunately, temperature independence of granular systems makes them no less complicated than fluids. The presence of dissipative forces like friction and deformation give rise to further complexity but also to a vast array of phenomena that make these systems very interesting. For example, in a vibrated granular system with particles of different sizes but equal density there are convective flows that lead to a size segregation of particles – and this is commonly referred to as the Brazil nut effect \cite{rosato, hong}. The bigger (heavier) particles are pushed to the top as the smaller particles get jostled around to fill voids. 

Even static granular systems exhibit behavior that defy expectations. One such example is the Janssen effect, named after the eponymous German engineer. In 1895 Janssen wrote down a mathematical description for the pressure at the bottom of a container filled with grains \cite{janssen_versuche_1895}. It was already known at the time that a silo full of corn husks for instance, does not obey Stevin's law which tells us that the pressure at the base due to a column of liquid is proportional to the height of liquid from the base to the surface. In the case of a static granular system the pressure at the base saturates with increasing filling height. This happens due to the formation of more robust force distribution networks that redirect the increased weight from added grains to the container side-walls rather than the base. This redistribution of the load is facilitated by grain-grain and grain-wall frictional forces. 

Chapter 2 will focus on an experiment carried out to study this Janssen effect in two dimensions. The grains in this study were placed on a horizontal moving surface such that they were compacted by frictional forces and not gravity as was the case for Janssen's original study. We show in this study that Janssen's relation does not hold in the case just described. Our study sheds light on how grain-wall friction is crucial in establishing the force network necessary to redistribute load to the side-walls. In the presence of a slipping surface in the friction compacted case, the grains adjacent to the walls fail to maintain static frictional contact so the entire system suddenly starts behaving like a fluid and it follows Stevin's law. We further show that it is possible to suppress this fluid-like behavior and re-establish Janssen's relation by introducing a suitable wall geometry so that the static friction is replaced by a suitable mechanical force that ultimately helps redirect increased load to the side-walls. The contents of Chapter 2 have been published as an article in Physical Review Letters \cite{karim_corwin}. 

Chapter 3 delves into the physics of obstructed granular flow. In this study we looked at the connection between obstacle geometry, granular shock front and lift forces on the obstacle. Whenever an object (e.g. a plane) moves through a medium (e.g. air) at a speed exceeding the speed of sound in it there is a shock front. As an example, with planes moving at less than the speed of sound, pressure waves are travelling faster than the plane itself so neighboring air molecules can rearrange in a continuous fashion. As the plane starts travelling at or faster than the speed of sound the “information” about the plane is no longer propagating fast enough. As a result the air around the plane become discontinuously compressed and this appears as a shock front. In air the angle of the cone forming the shock front depends on the speed of the object. Other factors such as the geometry of the bow and airfoils play a role in the complex shock waves around the plane. 

For Chapter 3, the system of an obstacle in flowing grains is analogous to a plane in air. For simplicity our experiment was conducted in a quasi-two dimensional system, meaning the thickness of the system is small enough that we can ignore it. Furthermore, the speed of sound in a low-density granular system is very low. This means an object moving at speeds of a few centimeters per second will experience a shock front. To vary the geometry in a systematic manner we generate the shapes from a set of curves defined by super-disks. This allows for the turning of a single knob to continuously vary geometry and study corresponding changes in the shock front and lift. 

The results of this study demonstrate that while granular materials exhibit fluid-like behavior they possess properties that are unique to these systems. For instance, we find that the shock front maintains the same shape even with varying geometry and impact speeds. We show that grains pile up around the obstacle such that the front separating the nearly static grains on the obstacle from flowing grains is described by a unique shape. The shape of this shock profile can be derived from force balance arguments with proper constraints. Our model for the shock front is a catenary – a function famously describing the shape of a chain hanging freely from two ends \cite{goldstein_classical_2002, mareno}.