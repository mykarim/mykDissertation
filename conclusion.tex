\chapter{Conclusion}

The unifying theme of this work is the physics of granular materials, both stationary and in a state of flow. We have further illustrated the ability of granular materials to exhibit solid-like and liquid-like behavior. We have detailed methods of tuning external conditions to preferentially retain one quality or the other. 

Chapter 1 focused on a well-known property of static granular systems concerning the pressure at the bottom of a container with grains. More than a hundred years ago, the Janssen equation described the property of contained granular systems to redirect increased load to the side walls. This means there is a saturation of pressure at the container base but an increase in load on side walls. In this work we demonstrated conditions necessary to sustain this phenomena. We also demonstrated how to tune the geometry of the system to recover the original Janssen effect. 

In contrast to Janssen's original experiment we constructed a two-dimensional granular system laid out horizontally, compacted by friction provided by a conveyor belt. We therefore removed the effect of gravity and could focus on the role of friction. We showed that with a constantly slipping surface and straight side-walls the system surprisingly behaves like a hydrostatic system following Stevin's law. In other words, Janssen's equation fails to capture the physics of the system. 

This vanishing of the Janssen effect is due to the slipping surface weakening frictional contacts between the grains and the side-walls. To recover the Janssen effect, we redesigned the side walls to consist of sawtooth patterns. The edges of the sawtooth were the same length scale as the grains to ensure enduring contacts. With the sawtooth walls there is a constant mechanical reaction force that replaces friction in a system with straight side walls. We found that even with the slipping surface of the conveyor belt this mechanical force is sufficient to enforce necessary conditions to redirect load to the side walls. Furthermore, this mechanical force can be tuned by the sawtooth angle and this determines the rate at which the bottom pressure saturates. This study illustrates the importance of friction in granular systems; the presence (or absence) of friction sets the onset of fluid-like properties. 

Chapter 3 focused on a different aspect of granular materials – specifically the characterization of shock fronts and consequent lift forces on obstructed granular flows. A quasi-two-dimensional hopper was constructed to provide steady dilute flow. The obstacle profile was derived from a family of curves known as super-disks. This enabled us to control its shape by tuning a single parameter – the exponent of the super-disk. We observed flows for several values of this exponent for a fixed obstacle size and granular impact speed. 

For all shape exponents the flow resulted in the formation of a quasi-static pile on the obstacle with a shock front separating this and the free flowing beads. We calculated flow fields by applying particle image velocimetry (PIV) and measured the lift force from a force sensor connected to the obstacle. To characterize the shock front and its effect on the flow field, we calculated the granular temperature (defined as the variance of the flow field), the flow field divergence and the shear-strain. Surprisingly we found a strong linear correlation between all three quantities near the shock front. 

We characterized the shock front as the local maxima of each measurement and showed that it is very well described by an inverted catenary. We derive the inverted catenary shape by applying simple force balance arguments and proper constraints. We find that unlike obstructed fluid flows, granular shock fronts maintain the same functional form across varying obstacle shapes and sizes and even impact speeds. This invariance in shock profile with respect to obstacle geometry is due to the quasi-static grains piling up to present the same shape. 

The functional form of the shock profile allows for accurate calculation of the lift force on the intruder. By calculating the force contributions from impact on the shock front as defined by catenary, momentum flux from quasi-static pile and horizontal component of the quasi-static load we can calculate the net lift on the obstacle. The agreement of the force calculation with direct force measurements is further evidence that the catenary description of the shock front is appropriate. 

This system constructed for studying granular flows has several avenues for further inquiry. In the current study the flow rate was kept constant for all measurements. It would be interesting to study the formation of the quasi-static pile by varying the flow rate. At very low flow rates the quasi-static pile would not exist. So what are the conditions necessary to form and sustain this static pile that is observed to present a constant profile (shock boundary) regardless of the intruder shape? The underlying physics governing this feature can be investigated using the same physical system. 

The system can also be easily modified to study drag and torque on the obstacle. Looking at the torque, for instance, may shed light on whether the quasi-static granular pile is minimizing the net torque on the obstacle.

%Besides uncovering some very interesting properties of granular materials the experiments described here are also illustrative of the value of simple table-top experiments.             

The experiments described above further illustrate how granular systems are replete with interesting phenomena that we do not fully understand. We know a little more when Janssen's equation fails and grains start to behave like liquids; and we know how to exploit geometry to retrieve Janssen's effect in the same system. We also show the tendency of obstructed grain flows to organize into predictably shaped piles that are surprisingly independent of obstacle shape, size and impact speed. While this work was driven by curiosity the results will hopefully extend our ability to predict the nature of granular matter that we find all around us. 