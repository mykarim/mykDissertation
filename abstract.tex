\abstract{	
The Janssen effect is a unique property of confined granular materials experiencing gravitational compaction in which the bottom pressure saturates with increasing filling height due to frictional interactions with side walls. By replacing gravitational compaction with frictional compaction on a horizontal conveyor belt,  we study friction-compacted 2D granular materials confined within fixed boundaries. Even with high-friction side walls the Janssen effect completely vanishes. Our results demonstrate that gravity-compacted granular systems are inherently different from friction-compacted systems in at least one important way: vibrations induced by sliding friction with the driving surface relax away tangential forces on the walls. Remarkably, we find that the Janssen effect can be recovered by replacing the straight side walls with a sawtooth pattern. The mechanical force introduced by varying the sawtooth angle $\theta$ can be viewed as equivalent to a tunable friction force. By construction, this mechanical friction force cannot be relaxed away by vibrations in the system. This work is described in Chapter II and has been published in \textit{Physical Review Letters}. \\

We experimentally study quasi-2d dilute granular flow around intruders whose shape, size and impact speeds are systematically varied. Direct measurement of the flow field reveals that three in-principle independent measurements of the non-uniformity of the flow field are in fact all linearly related: 1) granular temperature, 2) flow field divergence and 3) shear-strain rate.  The shock front is defined as the local maxima in each measurement. The shock front is well described by an inverted catenary and is driven by the formation of a dynamic arch during steady flow. We find universality in the functional form of the shock front within the range of experimental values probed. Changing the intruder size, concavity and impact speed only results in a scaling and shifting of the shock front. We independently measure the horizontal lift force on the intruder and find that it can be understood as a result of the interplay between the shock profile and intruder shape. 

This dissertation includes previously published and unpublished co-authored material.
	
%Need to end with this statement if you're working from papers published or soon to be published.
}